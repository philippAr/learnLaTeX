\documentclass[11pt]{beamer}
\author{Philipp Arras, Florian Nowak}
\title{\LaTeX -Kurs}
\date{11. Oktober 2014}

\usetheme{Berkeley}
%\setbeamercovered{transparent}
\setbeamertemplate{navigation symbols}{}
%\logo{}
%\institute{}
%\subject{}

\usepackage[utf8]{inputenc}
\usepackage[ngerman]{babel}
\usepackage{amsmath,amsfonts,amssymb}
\usepackage{graphicx,booktabs}


\begin{document}

%\begin{frame}
%\titlepage
%\end{frame}

%\begin{frame}
%\tableofcontents
%\end{frame}

\section{Fußnoten und Bibliographie}
\begin{frame}{Übung}
Sucht euch ein Buch oder einen Artikel aus; erstellt eine kurze \texttt{.bib}-Datei mit den entsprechenden Angaben. (Pflicht- und Wahlfelder der Grundtypen findet ihr in der \texttt{biblatex}-Doku ab Seite 7.) Schreibt einen kurzen Satz mit Verweis auf euer Buch bzw. euren Artikel und fügt diesem eine Fußnote hinzu.
\end{frame}

\end{document}