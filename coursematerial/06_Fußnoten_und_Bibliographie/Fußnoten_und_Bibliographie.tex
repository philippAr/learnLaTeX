\documentclass[11pt]{beamer}
\author{Philipp Arras, Florian Nowak}
\title{\LaTeX -Kurs}
\date{11. Oktober 2014}

\usetheme{Berkeley}
%\setbeamercovered{transparent}
\setbeamertemplate{navigation symbols}{}
%\logo{}
%\institute{}
%\subject{}

\usepackage[utf8]{inputenc}
\usepackage[ngerman]{babel}
\usepackage{amsmath,amsfonts,amssymb}
\usepackage{graphicx,booktabs}


\begin{document}

%\begin{frame}
%\titlepage
%\end{frame}

%\begin{frame}
%\tableofcontents
%\end{frame}

\section{Fußnoten und Bibliographie}
\begin{frame}[fragile]{Fußnoten}
\begin{itemize}
\item Zusätzlicher Text, der nicht ins Hauptdokument passt\\
(für besseren Textfluss)
\item \verb~\footnote{...}~
\end{itemize}
\end{frame}

\begin{frame}{Erstellen von Bibliographien}
\begin{itemize}
\item Verweise auf Literatur sollten weitestgehend automatisiert werden
\item[$\Rightarrow$] Vermeidet Fehler
\end{itemize}
\end{frame}

\begin{frame}{Erstellen von Bibliographien}
\begin{itemize}
\item Verweise auf Literatur sollten weitestgehend automatisiert werden
\item Anordnen und Sortieren kann externern Programmen und Paketen überlassen werden
\item Externe Programme basieren auf (selbst erstellter) Datenbank
\item Bib{\TeX} bzw. \texttt{biber} \emph{sortieren} Einträge
\item Pakete wie \texttt{biblatex} oder \texttt{natbib} \emph{formatieren} die Literaturangaben und -verzeichnisse
\end{itemize}
\end{frame}

\begin{frame}{Beispiel Bibliographie}
\emph{(siehe }\texttt{exmpl-biber.pdf}\emph{)}
\end{frame}

\begin{frame}[fragile]{Literaturverweise ohne externe Hilfe}
\begin{itemize}
\item Intern mittels \texttt{thebibliography}\\
(am Ende des Dokuments):
\end{itemize}
\begin{verbatim}
...

\begin{thebibliography}{widest entry}
\bibitem[GK06]{GreeneKrantz} R. Greene und S. %
  Krantz. \textit{Function Theory of one Complex %
  Variable}. 3. ed. AMS, 2006.
\bibitem[Kas14]{Kasten} ...
\bibitem[RS00]{RemmertSchumacher} ...
\end{thebibliography}

\end{document}
\end{verbatim}
\end{frame}

\begin{frame}[fragile]{Literaturverweise ohne externe Hilfe}
\begin{verbatim}
\begin{thebibliography}{GK06}
\bibitem[GK06]{GreeneKrantz} R. Greene und S. %
  Krantz. \textit{Function Theory of one Complex %
  Variable}. 3. ed. AMS, 2006.
\end{thebibliography}

\end{document}
\end{verbatim}
\begin{itemize}
\item \verb~\cite{...}~ setzt (automatisierten) Literaturverweis\\
Beispiel: \emph{siehe }\texttt{exmpl-thebibliography.pdf}
\end{itemize}
\end{frame}

\begin{frame}[fragile]{Literaturverweise ohne externe Hilfe}
\begin{verbatim}
\begin{thebibliography}{GK06}
\bibitem[GK06]{GreeneKrantz} R. Greene und S. %
  Krantz. \textit{Function Theory of one Complex %
  Variable}. 3. ed. AMS, 2006.
\end{thebibliography}

\end{document}
\end{verbatim}
\begin{itemize}
\item \verb~\cite{...}~ setzt (automatisierten) Literaturverweis\\
\item[$-$] Nicht wirklich elegant
\end{itemize}
\end{frame}

\begin{frame}[fragile]{Literaturverweise mit externer Hilfe}
\begin{itemize}
\item Separate \texttt{.bib}-Datei:
\end{itemize}
\begin{verbatim}
% tinybib.bib

@Book{GreeneKrantz,
 author    ={Greene, R. and Krantz, S.},
 title     ={Function Theory of one Complex %
   Variable},
 year      ={2006},
 publisher ={AMS},
 edition   ={3. ed.}}
                                                       
...
\end{verbatim}
\end{frame}

\begin{frame}{Literaturverweise mit externer Hilfe}
\begin{itemize}
\item Separate \texttt{.bib}-Datei
\item[$+$] Programm Bib{\TeX} übernimmt Sortierung und Verwaltung der Einträge
\item \texttt{biber} ist als neuste Bib{\TeX}-Weiterentwicklung utf8-fähig und sehr flexibel einsetzbar – allerdings nur mit Paket \texttt{biblatex}
\end{itemize}
\end{frame}

\begin{frame}{Literaturverweise mit externer Hilfe}
\begin{itemize}
\item Separate \texttt{.bib}-Datei
\item[$+$] Programm Bib{\TeX} übernimmt Sortierung und Verwaltung der Einträge
\item \texttt{biber} ist als neuste Bib{\TeX}-Weiterentwicklung utf8-fähig und sehr flexibel einsetzbar – allerdings nur mit Paket \texttt{biblatex}
\item[$+$] (Fast) fertige Bib{\TeX}-Einträge können über \url{http://arxiv.org/} oder \url{http://scholar.google.com/} oder auch HEIDI generiert werden
\end{itemize}
\end{frame}

\begin{frame}{biber in Aktion}
\emph{(siehe }\texttt{exmpl-biber.tex}\emph{ und }\texttt{tiny.bib}\emph{)}
\end{frame}

\begin{frame}{Übung}
Sucht euch ein Buch oder einen Artikel aus; erstellt eine kurze \texttt{.bib}-Datei mit den entsprechenden Angaben. (Pflicht- und Wahlfelder der Grundtypen findet ihr in der \texttt{biblatex}-Doku ab Seite 7.) Schreibt einen kurzen Satz mit Verweis auf euer Buch bzw. euren Artikel und fügt diesem eine Fußnote hinzu.
\end{frame}

\end{document}