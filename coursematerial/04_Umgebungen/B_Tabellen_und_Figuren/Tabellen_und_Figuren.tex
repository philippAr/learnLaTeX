\documentclass[11pt]{beamer}
\author{Philipp Arras, Florian Nowak}
\title{\LaTeX -Kurs}
\date{11. Oktober 2014}

\usetheme{Berkeley}
%\setbeamercovered{transparent}
\setbeamertemplate{navigation symbols}{}
%\logo{}
%\institute{}
%\subject{}

\usepackage[utf8]{inputenc}
\usepackage[ngerman]{babel}
\usepackage{amsmath,amsfonts,amssymb}
\usepackage{graphicx,booktabs}


\begin{document}

%\begin{frame}
%\titlepage
%\end{frame}

%\begin{frame}
%\tableofcontents
%\end{frame}

\section{Tabellen und Figuren}
\begin{frame}{Übungen}
\begin{enumerate}
\item Erstellt eine kleine (schöne) Tabelle mit Inhalt eurer Wahl; verwendet dabei \emph{umbedingt} das Paket \texttt{booktabs}! Fasst eure Tabelle in eine \texttt{table}-Umgebung ein und gebt ihr eine Unterschrift. Schreibt einen kurzen Satz (darunter) mit Verweis auf die Tabelle.
\item 
\end{enumerate}
\end{frame}


\end{document}