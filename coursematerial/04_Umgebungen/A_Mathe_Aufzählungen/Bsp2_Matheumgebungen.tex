\documentclass[11pt,a4paper]{scrartcl}
\usepackage[utf8]{inputenc}
\usepackage[german]{babel}
\usepackage{amsmath}
\usepackage{amsfonts}
\usepackage{amssymb}
\usepackage{graphicx, blindtext}
\usepackage[left=2cm,right=2cm,top=2cm,bottom=2cm]{geometry}
\author{Author}
\title{Titel}
\begin{document}

\section{Alignment}

\begin{align}
x&=y       & X&=Y  & a&=b+c\\
x' &= y'     & X'&=Y' & a'&= b\\
x+x'&=y+y' & X+X'&=Y+Y' & a'b&=c'b
\end{align}

\section{Unterschiedliche Umgebungen}
Das ist equation*.
\begin{equation*}
a=b
\end{equation*}
Das ist equation.
\begin{equation}
a=b
\end{equation}
Das ist equation mit split.
\begin{equation}\label{xx}
\begin{split}
a& =b+c-d\\
 & \quad +e-f\\
 & =g+h\\
 & =i
\end{split}
\end{equation}
Das ist multiline.
\begin{multline}
a+b+c+d+e+f\\
+i+j+k+l+m+n
\end{multline}
Das ist gather.
\begin{gather}
a_1=b_1+c_1\\
a_2=b_2+c_2-d_2+e_2
\end{gather}
Das ist align.
\begin{align}
a_1& =b_1+c_1\\
a_2& =b_2+c_2-d_2+e_2
\end{align}
Das ist nochmal align.
\begin{align}
a_{11}& =b_{11}&a_{12}& =b_{12}\\
a_{21}& =b_{21}&a_{22}& =b_{22}+c_{22}
\end{align}
Das ist flalign*.
\begin{flalign*}
a_{11}& =b_{11} & a_{12}& =b_{12}\\
a_{21}& =b_{21} & a_{22}& =b_{22}+c_{22}
\end{flalign*}

\end{document}
