\documentclass[11pt,a4paper]{scrartcl}
\usepackage[utf8]{inputenc}
\usepackage[german]{babel}
\usepackage{amsmath}
\usepackage{amsfonts}
\usepackage{amssymb}
\usepackage{graphicx, cases}
\usepackage{blindtext,hyperref}
\usepackage[left=2cm,right=2cm,top=2cm,bottom=2cm]{geometry}
\author{Philipp Arras}

%_______________________________________________________________________________
% Listings
%_______________________________________________________________________________
\usepackage{listings}
\usepackage{color}

\definecolor{mygreen}{rgb}{0,0.6,0}
\definecolor{mygray}{rgb}{0.5,0.5,0.5}
\definecolor{mymauve}{rgb}{0.58,0,0.82}

\lstset{ %
%  backgroundcolor=\color{white},   % choose the background color; you must add \usepackage{color} or \usepackage{xcolor}
basicstyle=\footnotesize,        % the size of the fonts that are used for the codew
breakatwhitespace=false,         % sets if automatic breaks should only happen at whitespace
breaklines=true,                 % sets automatic line breaking
  captionpos=b,                    % sets the caption-position to bottom
  commentstyle=\color{mygreen},    % comment style
%  deletekeywords={...},            % if you want to delete keywords from the given language
%  escapeinside={\%*}{*)},          % if you want to add LaTeX within your code
%  extendedchars=true,              % lets you use non-ASCII characters; for 8-bits encodings only, does not work with UTF-8
  frame=single,                    % adds a frame around the code
%  keepspaces=true,                 % keeps spaces in text, useful for keeping indentation of code (possibly needs columns=flexible)
  keywordstyle=\color{blue},       % keyword style
  language=TeX,                 % the language of the code
%  morekeywords={*,...},            % if you want to add more keywords to the set
  numbers=left,                    % where to put the line-numbers; possible values are (none, left, right)
  numbersep=5pt,                   % how far the line-numbers are from the code
  numberstyle=\tiny\color{mygray}, % the style that is used for the line-numbers
  rulecolor=\color{black},         % if not set, the frame-color may be changed on line-breaks within not-black text (e.g. comments (green here))
%  showspaces=false,                % show spaces everywhere adding particular underscores; it overrides 'showstringspaces'
%  showstringspaces=false,          % underline spaces within strings only
%  showtabs=false,                  % show tabs within strings adding particular underscores
  stepnumber=2,                    % the step between two line-numbers. If it's 1, each line will be numbered
  stringstyle=\color{mymauve},     % string literal style
  tabsize=2,                       % sets default tabsize to 2 spaces
  title=\lstname                   % show the filename of files included with \lstinputlisting; also try caption instead of title
}



\title{\LaTeX -Kurs: 04A Umgebungen}
\subtitle{Mathe und Aufzählungen}
\begin{document}
\maketitle

\section{Allgemein}
\begin{enumerate}
\item Umgebungen sind Makros, die auf einen begrenzten Textbereich wirken.
\item Grundsätzlich gilt: \LaTeX -Befehle beginnen mit einem $\backslash$ und enden mit einem Leerzeichen. Das kann manchmal Probleme machen: \verb|\LaTeX hilft| gibt \LaTeX hilft.\\
Die Lösung ist \verb|\LaTeX\ hilft|. Dies gibt \LaTeX\ hilft.
\item Jetzt aber zu Umgebungen: Sie haben einen Anfang und ein Ende:
	\begin{enumerate}
	\item \verb|\begin{...}| und \verb|\end{...}|: Definiert Anfang und Ende
	\item Manche Umgebungen gelten bis zum nächsten ähnlichen Befehl, wie z.B. $\backslash$ Huge 
    \lstinputlisting{Bsp1_Allgemein.tex}
	\end{enumerate} 
\item Die wichtigsten Umgebungen:
  \begin{enumerate}
  \item Wir kennen schon \emph{document}.
  \item Textformatierungen (\emph{center}, \emph{flushleft} und \emph{flushright} )
  \item Zitate (\emph{quote} und \emph{quoatation})
  \item Listen (\emph{itemize} und \emph{enumerate})
  \item Direkte Ausgabe (\emph{verbatim} und \emph{verb})
  \item Abbildungen (\emph{figure})
  \item Sehr nützlich: \emph{minipage} (selber nachgucken)
  \item Für Tabellen: \emph{tabular} 
  \item Für Formeln: \emph{align} 
  \end{enumerate}
\item Wir konzentrieren uns auf Mathe-Umgebungen und Aufzählungen
\end{enumerate}

\section{Aufzählungen}
Hier einfach nur kurz itemize und enumerate erklären. Und description:
\begin{description}
  \item[First] \hfill \\
  The first item
  \item[Second] \hfill \\
  The second item
  \item[Third] \hfill \\
  The third etc \ldots
\end{description}

\section{Mathe-Umgebungen}
\begin{itemize}
\item Formeln stehen im Fließtext und abgesetzt. Abgesetzte Formeln haben eine Nummerierung oder auch nicht.
\item Formeln im Fließtext mit \verb|$ ... $|.
\item abgesetzte Formeln mit \verb|\begin{ailgn} ... \end{align}| und \verb|\begin{ailgn*} ... \end{align*}|. Mit \verb|\\| macht man eine neue Zeile und mit \verb|\nonumber| bekommt die aktuelle Formel keine Nummer.
\item Verwende die Pakete amsmath (bessere Umgebungen, Dokumentation: \url{ftp://ftp.ams.org/ams/doc/amsmath/amsldoc.pdf}), amsfonts (fraktur, mehr Zeichen, ...) und amssymb (mehr Zeichen \url{http://milde.users.sourceforge.net/LUCR/Math/mathpackages/amssymb-symbols.pdf})
\item \lstinputlisting{Bsp2_Matheumgebungen.tex}
\item Indices (auch verschachelt [System erklären, dass immer nur ein Zeichen als Index genommen wird], Brüche, Summen, Produkte (prod), cdot, ldot(s), Integral, Wurzeln, Matrizen
\item Sonderzeichen \verb|\(1 \le 2 \ge 0 \neq 4\), \quad \(1 \ll 10^{20} \gg 10^{-5} \pm \mp\)|:
  \begin{align*}
  1 \le 2 \ge 0 \neq 4, \quad 1 \ll 10^{20} \gg 10^{-5} \pm \mp
  \end{align*}
\item Matrizen (\url{http://www.kkittel.de/wiki/doku.php?id=mathematik:matrizen})
\item griechische Buchstaben
\item Formatiere mathematischen Text
  \begin{itemize}
  \item mathrm: für Einheiten, ...
  \item mathit: kursiv (Zahlen auch!)
  \item mathbf: für Vektoren
  \item mathfrak: Fraktur z.B. für Lie-Algebren
  \item mathcal
  \item mathbb
  \item mathscr (braucht mathrsfs-Package)
  \end{itemize}
\item Mehrere Fälle (benötigt das Paket cases):
\begin{verbatim}
\begin{numcases}{E = mc^2}
m \neq 0 & Masselose Teilchen\\
m < 0 & Antiteilchen (?)\\
m > 0 & normale Teilchen
\end{numcases}
\end{verbatim}
\begin{numcases}{E = mc^2}
m \neq 0 & Masselose Teilchen\\
m < 0 & Antiteilchen (?)\\
m > 0 & normale Teilchen
\end{numcases}
\item Text in Matheumgebung mit \verb|\text{ ... }|.
\item Abstände mit Komma, Semikolon, quad, qquad oder hspace
\item Operatorennamen: $\arccos \arcsin \arg \cos \cot \coth \deg \det \exp \gcd \inf \injlim \lg \lim \limsup \ln \max$ \\ $\min \projlim \sec \sinh \sup \tanh$ mit Backslash
\item Große Klammern:
\[(\frac{\int^a x dx}{\sum_{n=1} x})\]
Vs. 
\[ \left( \frac{\int^a x dx}{\sum_{n =1} x} \right) \]

\end{itemize} 

Alles Wichtige steht hier: \url{http://en.wikibooks.org/wiki/LaTeX/Mathematics#Matrices_in_running_text} und im Repository: 05-Mathesatz I.pdf


Aufgabe: Suche dir einen Abschnitt aus dem Seminarvortrag von Philipp und texe es nach. Alternativ natürlich auch gerne eine eigene Rechnung etc.


\end{document}
