\documentclass[a4paper,ngerman]{scrartcl} % 11pt voreingestellt bei scrbook, scrreprt sowie scrartcl; a4paper evtl. überflüssig: jenachdem, was bei der TeX Live Installation als Standard gesetzt wurde
\author{Philipp Arras}
\title{Umgebungen}
%\date{Datum}

\usepackage[utf8]{inputenc}
\usepackage{babel} % ngerman bereits oben gewählt
\usepackage{amsmath,amsfonts,amssymb,amsthm}
\usepackage{graphicx}
\usepackage[left=2cm,right=2cm,top=2cm,bottom=2cm]{geometry} % Ginge auch mit margin=2cm

\usepackage{blindtext}

\begin{document}
\maketitle

\section{Aufzählungen}

\subsection{Die Itemize-Umgebung}

\blindtext
\begin{itemize}
\item Das ist unser erster Punkt.
\item Und noch einer.
\end{itemize}
\blindtext


\subsection{Die Enumerate-Umgebung}
\blindtext
\begin{enumerate}
\item Jetzt machen wir eine nummerierte Liste.
\item Der zweite Eintrag.
\end{enumerate}
\blindtext

\subsection{Verschachtelte Umgebungen}

\begin{itemize}
\item Hallo
	\begin{enumerate}
	\item Enumerate Umgebugn
	\item Hallo
	\item World
	\end{enumerate}
\item dsfjkl
\item jkjkkjkj
\end{itemize}


\section{Matheumgebungen}

\subsection{Im Fließtext}

Was wir schon immer mal lernen wollten, war der Satz des Pythagoras: $a^2+b^2=c^2$. Und das war es auch schon. Aber vielleicht wollen wir eine Formel auch mal ein bisschen größer sehen:
\begin{align}
\int x^2 dx = \frac{x^3}{3} 
\end{align}
Und danach geht es mit dem Text weiter.
\blindtext
\begin{align}
\int \sin (x) dx = - \cos(x)
\end{align}

\subsection{Ohne Nummerierung}

Was wir schon immer mal lernen wollten, war der Satz des Pythagoras: $a^2+b^2=c^2$. Und das war es auch schon. Aber vielleicht wollen wir eine Formel auch mal ein bisschen größer sehen:
\begin{align*}
\int x^2 dx = \frac{x^3}{3} 
\end{align*}
Und danach geht es mit dem Text weiter.
\blindtext
\begin{align*}
\int \sin (x) dx = - \cos(x)
\end{align*}

\subsection{Indices}
\begin{align}
\sum_{n=1}^\infty a_n = \frac{\pi}{6}
\end{align}


\subsection{Die Align-Umgebung in Aktion}
\begin{align}
(x-1)(x-2) &= x^2-x-2x+2\\
&= x^2-3x+2
\end{align} 

\subsection{Eine Liste von Mathe-Befehlen}
\begin{align}
1 \leq 2 , 3 \geq 1\\
\pm \\
0\neq 1
\end{align}

Die Zahlen:
\begin{align}
\mathbb{R}, \mathbb{Q}, \mathbb{N}, \mathbb{Z}
\end{align}

\begin{align}
\mathfrak{a}
\end{align}

Indices verschalten

Griechische Buchstaben: $\alpha \beta \gamma \Gamma$

\begin{align}
\left(
   \begin{array}{ccc}
     a_{11} & \cdots & a_{1n} \\
     \vdots & \ddots & \vdots \\
     a_{m1} & \cdots & a_{mn}
   \end{array}
\right)
\end{align}






\end{document}
