\documentclass[11pt,a4paper]{scrreprt}
\usepackage[utf8]{inputenc}

\usepackage{mdframed}
\definecolor{examplesBackground}{RGB}{220,226.5,237}
\definecolor{examplesHeaderLine}{RGB}{122,150,191}
\definecolor{claimsHeaderLine}{gray}{0.97}

\newmdenv[
	topline=false, bottomline=false, leftline=false, rightline=false,
	backgroundcolor=examplesBackground,
	frametitle={\tiny\hspace{2cm}}, frametitlebackgroundcolor=examplesHeaderLine,
	frametitleaboveskip=-1pt]
	{abstractSection}

\begin{document}
\begin{abstractSection}
Algebraic geometry heavily relies on the fact that there is a strong connection between ideals being algebraic objects and varieties being geometric ones. We will start with the basic definitions and arrive via \emph{Hilbert's Nullstellensatz} at the Algebra-Geometry dictionary. Already at this point, I would like to stress that it holds only for ideals in polynomial rings over algebraically closed fields. Without this assumption algebraic geometry is far from being intuitive and descriptive.

After defining an adequate topology we decompose varieties uniquely into irreducible ones. Unfortunately, this does not translate into the language of ideals in full generality: Only radical ideals can be decomposed into prime ideals uniquely. For non-radical ideals we need a more sophisticated concept: the primary decomposition.
\end{abstractSection}

\end{document}