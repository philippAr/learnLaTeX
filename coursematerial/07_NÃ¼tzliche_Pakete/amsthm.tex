\documentclass[11pt,a4paper]{scrreprt}
\usepackage[utf8]{inputenc}
\usepackage{amsmath,amsfonts,amssymb, amsthm}

\usepackage[backend=biber,style=alphabetic]{biblatex}

\newtheorem{defn}{Definition}
\newtheorem{thm}[defn]{Theorem}
\newtheorem{lem}[defn]{Lemma}
\newtheorem{prop}[defn]{Proposition}
\newtheorem{cor}[defn]{Corollary}
\newtheorem{ex}[defn]{Example}

\newcommand{\R}{\mathbb{R}}
\newcommand{\C}{\mathbb{C}}
\newcommand{\N}{\mathbb{N}}
\newcommand{\Q}{\mathbb{Q}}
\newcommand{\Zn}{\mathbb{Z}^n_{\geq 0}}
\newcommand{\V}{\mathbf{V}}
\newcommand{\I}{\mathbf{I}}
\newcommand{\mvar}[2]{#1_1,\ldots , #1_{#2}}
\newcommand{\kxn}{k[\mvar{x}{n}]}
\newcommand{\fs}{\mvar{f}{s}}
%\newcommand{\mc}[1]{\boldsymbol{1}^{(#1)}}
\newcommand\mc[1]{{\vphantom{\bar{\boldsymbol{1}}}\boldsymbol{1}}^{\mkern-2mu (#1)}}
\newcommand{\mcb}[1]{\bar{\boldsymbol{1}}^{(#1)}}
\newcommand{\ya}{\mc{1}\mcb{4}\mc{5}}
\newcommand{\yb}{\mc{2}\mcb{3}\mc{5}}
\newcommand{\yc}{\mc{2}\mcb{4}\mc{6}}
\newcommand{\yd}{\mc{1}\mc{2}\mcb{6}}
\newcommand{\ye}{\mcb{3}\mc{4}\mc{6}}
\newcommand{\yf}{\mcb{5}\mc{6}}
\newcommand{\yfs}{\mcb{5}\mc{6}\mc{6}}

\setlength{\parindent}{0mm}

\begin{document}
\begin{defn}
Let $R$ be a ring. Then $R[X]$ is the set of all finite sequences 
\begin{align*}
R^{(\N_0)} := 
\left\lbrace (a_i)_{i\in\N_0} \; : \;  a_i \in R, a_i=0 \text{ for almost all } i \right\rbrace.
\end{align*}
We define an addition component-by-component ($(a_i)_{i\in \N_0} + (b_i)_{i\in \N_0} := (a_i + b_i)_{i\in \N_0}$) and a multiplication by the convolution of both sequences ($(a_i)_{i\in \N_0} \cdot (b_i)_{i\in \N_0} := (\sum_{i_0}^k) a_i b_{k-i}$).

The resulting ring is denoted by $R[X]$ and called \emph{polynomial ring over $R$}. Let us define $x := (0,1,0,0,\ldots)$. Then 
\begin{align*}
(\underbrace{0, \ldots , 0,}_{k \text{ zeros}} 1, 0, \ldots ) \equiv x^k = \underbrace{x \cdot \ldots \cdot x}_{k\text{ times}}
\end{align*}
and all Elements $f\in R^{(\N_0)}$ can be written as
\begin{align*}
f = a_0 + a_1 x + a_2 x^2 + \cdots + a_n x^n.
\end{align*}
\end{defn}

\begin{defn}
Let $R$ be a ring. The \emph{polynomial ring in $n$ variables over $R$} is defined recursively:
\begin{align*}
R[\mvar{x}{n}] := R[\mvar{x}{n-1}] [x_n].
\end{align*}
The elements of $R[\mvar{x}{n}] $ can be written as 
\begin{align*}
\sum_{k = (\mvar{k}{n}) \in \N^n} a_k \: x_1^{k_1} \cdots x_n^{k_n}.
\end{align*}
\end{defn}

In the sequel, we will only consider polynomial rings over fields, e.g. $\R$ or $\C$. To avoid endless repetitions let $k$ be a field from now on. We move on to the basic algebraic object of this work.

\begin{defn}
A subset $I\subset k[\mvar{x}{n}]$ is an \emph{ideal} if:
\begin{enumerate}
\item $0 \in I$.
\item $f,g\in I \;\Rightarrow \; f+g\in I$.
\item $f\in I$ and $h\in k[\mvar{x}{n}] \;\Rightarrow\; hf\in I$.
\end{enumerate}
\end{defn}


\begin{lem}
If $\mvar{f}{s}\in k[\mvar{x}{n}]$, then 
\begin{align*}
\langle \mvar{f}{s} \rangle := \left\lbrace \sum_{i=1}^s h_i f_i \; : \;  \mvar{h}{s} \in \kxn \right\rbrace
\end{align*}
 is an ideal of $\kxn$ and is called the \emph{ideal generated by the $f_i$s}.
\end{lem}

This fact leads to the first interesting observation. When manipulating systems of linear equations one is allowed to multiply any equation by a polynomial and to add two equations. These rules correspond to the definition of an ideal. Therefore we can think of the ideal $\langle \mvar{f}{s} \rangle$ as the set of all consequences of the equations $f_1=f_2=\ldots =f_s=0$ in the sense that if all generating polynomials are set to zero then all elements of the ideal will be zero as well.

We now come to special ideals which will play an important role in the context of varieties.

\begin{defn}
An ideal $I$ is \emph{radical} if the following implication holds:
\begin{align*}
f^m \in I \text{ for some integer } m\geq 1 \quad \Rightarrow\quad f\in I.
\end{align*}
\end{defn}

Given an arbitrary ideal it is always possible to produce a radical ideal.
\begin{defn}
Let $I \subset \kxn$ be an ideal. The \emph{radical} of $I$  is 
\begin{align*}
\sqrt{I} := \left\lbrace  f\; :\; f^m \in I \text{ for some integer } m\geq 1  \right\rbrace .
\end{align*}
\end{defn} 

\begin{lem}
\label{lemradideals}
$\sqrt{I}$ is a radical ideal and $I\subset\sqrt{I}$.
\end{lem}
\begin{proof}
This is easy to see.
\end{proof}
To see that equality is not guaranteed consider $I= \langle x^2 \rangle \subset \C [x]$. Then $\sqrt{I} = \langle  x\rangle \neq \langle x^2 \rangle $.

We move on to another important property an ideal can have.
\begin{defn}
An ideal $I \subset \kxn $ is \emph{prime} if the following implication is true:
\begin{align*}
f, g \in \kxn \text{ and } fg\in I \quad \Rightarrow \quad f\in I \text{ or } g \in I.
\end{align*}
\end{defn}
\end{document}