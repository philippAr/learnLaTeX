\documentclass[11pt]{beamer}
\usetheme{Berkeley}
\usepackage[utf8]{inputenc}
\usepackage[german]{babel}
\usepackage{amsmath}
\usepackage{amsfonts}
\usepackage{amssymb}
\usepackage{graphicx}
\author{Philipp Arras, Florian Nowak}
\title{\LaTeX -Kurs}
%\setbeamercovered{transparent} 
\setbeamertemplate{navigation symbols}{} 
%\logo{} 
%\institute{} 
\date{11. Oktober 2014} 
%\subject{} 
\begin{document}

\begin{frame}
\titlepage
\end{frame}

%\begin{frame}
%\tableofcontents
%\end{frame}

\section{Gliederung}
\begin{frame}{Gliederung}
\begin{enumerate}
\item Konzept, Vor- und Nachteile, Geschichte, Editoren
\item Einführung: Unsere ersten LaTeX-Dokumente
%	\begin{itemize}
%	\item Unser erstes LaTeX-Dokument
%	\item logische Gliederung eines LaTeX-Dokuments
%	\item Standardpakete
%	\item Kopf- und Fußzeilen
%	\end{itemize}
\item Umgebungen
%	\begin{itemize}
%	\item Matheumgebungen
%	\item Aufzählungen
%	\item Tabellen
%	\item Fußnoten
%	\item Figuren
%	\end{itemize}
\item Wie mache ich ein großes Dokument übersichtlich?
\item Bibliographie mit biber
\item nützliche Pakete (setspace, csquotes, amsthm, mdframed, listings)
\item weitere Pakete
%	\begin{itemize}
%	\item Beamer
%	\item TikZ
%	\end{itemize}
\end{enumerate}
\end{frame}

\section{Zeiltplan}
\begin{frame}{Zeitplan}
\begin{tabular}{c|l|c}
10:15 - 10:30 & Geschichte und Konzept & F\\
10:30 - 11:30 & Unsere ersten zwei \LaTeX -Dokumente & P\\
11:30 - 12:30 & Umgebungen & P\\
12:30 - 13:00 & Umgebungen & F\\
\multicolumn{3}{c}{\textit{Mittagspause}}\\
14:00 - 15:00 & Übersichtlichkeit und Bibliographie & F\\
15:00 - 15:30 & Nützliche Pakete & P\\
\multicolumn{3}{c}{\textit{Pause}}\\
16:00 - 17:00 & Beamer und TikZ & P
\end{tabular}
\end{frame}


\end{document}