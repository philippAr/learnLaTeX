\documentclass[a4paper,ngerman]{scrartcl} % 11pt voreingestellt bei scrbook, scrreprt sowie scrartcl; a4paper evtl. überflüssig: jenachdem, was bei der TeX Live Installation als Standard gesetzt wurde
\author{Philipp Arras}
\title{\LaTeX -Kurs: 05 1 GitHub}
\date{Datum}

\usepackage[utf8]{inputenc}
\usepackage{babel} % ngerman bereits oben gewählt
\usepackage{amsmath,amsfonts,amssymb,amsthm}
\usepackage{graphicx}
\usepackage[left=2cm,right=2cm,top=2cm,bottom=2cm]{geometry} % Ginge auch mit margin=2cm

\usepackage{blindtext}

\begin{document}
\maketitle
Git ist ein Versionsverwaltungsprogramm.

\begin{enumerate}
\item sudo apt-get install git
\item Account bei github erstellen
\item mkdir ~/repos
	In diesen Ordner kommen alle deine Git-Projekte rein
\item cd ~/repos
\item git clone https://github.com/philippAr/learnLatex
	Mit diesem Befehl lädst du die aktuelle Version des Repositories herunter
\item An mich eine Mail schreiben und mir deinen Benutzernamen mitteilen. Ich muss dir dann die Berechtigung geben, dass du auf dem Projekt Änderungen machen kannst.
\item git status
	Das ist ein sehr nützlicher Befehl. Den solltest du zwischen den nächsten Schritten immer ausführen um zu sehen, was passiert
\item Editiere eine Datei, z.B. diese, und speichere sie
\item git add introductionGit.txt
	mit diesem Befehl wird die neue Version der Datei gestaged. Das ist ein Zwischenschritt, der für Anfänger überflüssig erscheint, später aber sehr hilfreich sein kann. Alternativ kann man ins Hauptverzeichnis wechseln (cd ~/repos/learnLatex) und git add -A ausführen. Dann werden alle neuen  und veränderten Dateien gestaged
\item git commit -m "Hier muss eine Nachricht hin, die beschreibt, was geändert wurde."
	Mit diesem Befehl werden alle gestageten Dateien commitet, d.h. der "Geschichtsschreibung" hinzugefügt. Ab jetzt hast du die Möglichkeit später genau diesen Zustand deines ganzen Repositories wiederherzustellen
\item git push origin master
	Hiermit lädst du alle Änderungen zu GitHub hoch. Es ist nicht notwendig nach jedem Commit ein Push auszuführen. Wenn du mehrere Commits hintereinander machst, weil du z.B. gerade unterwegs keine Internetverbindung hast, läd der push-Befehl alle Commits automatisch auf einmal hoch.
\item git pull origin master
	Hiermit lädst du alle Änderungen von GitHub herunter. Das sollte man also immer ausführen, bevor man anfängt zu arbeiten.
\end{enumerate}



Außerdem wichtig ist:
\begin{itemize}
\item Dateien kopieren, verschieben, löschen immer mit git cp, git mv, git rm. Sonst bekommt git das ja nicht mit.
\item Es könnte ja theoretisch sein (wird bei uns wahrscheinlich nicht passieren), dass wir beide die gleiche Datei bearbeiten. Angenommen ich pushe zuerst und dann du. Dann wirst du eine Fehlermeldung bekommen und müsstest einen merge ausführen. Das ist nicht so einfach. Aber wie gesagt: bei unserem kleinen Projekt sehr unwahrscheinlich
\item Es gibt viele GUIs für git. Diese eigenen sich sehr gut um sich die History anzeigen zu lassen (git log ist sehr unübersichtlich^^) aber nicht gut um die oben genannten Schritte auszuführen. Dann weiß man nicht, was unter der Oberfläche passiert. Also lieber aus der Shell heraus bedienen.
\end{itemize}

Git ist zuerst ein bisschen kompliziert zu verstehen, bringt dann aber riesige Vorteile mit sich.


\end{document}
