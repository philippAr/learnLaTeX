\documentclass{beamer}
\author{Philipp Arras, Florian Nowak}
\title{\LaTeX -Kurs}
\date{11. Oktober 2014}

\usetheme{Berkeley}
%\setbeamercovered{transparent}
\setbeamertemplate{navigation symbols}{}
%\logo{}
%\institute{}
%\subject{}

\usepackage[utf8]{inputenc}
\usepackage[ngerman]{babel}
\usepackage{amsmath,amsfonts,amssymb}
\usepackage{graphicx,booktabs}


\begin{document}

\begin{frame}
\titlepage
\end{frame}

\section*{Outline}
\begin{frame}
\tableofcontents
\end{frame}

\section{Abschnitt 1}
\begin{frame}{Folie 1}
\begin{enumerate}
\item Erstens
\item Zweitens
\item Drittens
\end{enumerate}
\end{frame}

\section{Primzahlen}
\begin{frame}
  \frametitle{There Is No Largest Prime Number}
  \framesubtitle{The proof uses \textit{reductio ad absurdum}.}
  \begin{theorem}
    There is no largest prime number.
  \end{theorem}
  \begin{proof}
    \begin{enumerate}
    \item<1-> Suppose $p$ were the largest prime number.
    \item<2-> Let $q$ be the product of the first $p$ numbers.
    \item<3-> Then $q + 1$ is not divisible by any of them.
    \item<1-> But $q + 1$ is greater than $1$, thus divisible by some prime
      number not in the first $p$ numbers.\qedhere
    \end{enumerate}
\end{proof}
  \uncover<4->{The proof used \textit{reductio ad absurdum}.}
\end{frame}

\begin{frame}
  \frametitle{What's Still To Do?}
  \begin{block}{Answered Questions}
    How many primes are there?
  \end{block}
  \begin{block}{Open Questions}
    Is every even number the sum of two primes?
  \end{block}
\end{frame}

\begin{frame}
  \frametitle{What's Still To Do?}
  \begin{columns}
    \column{.5\textwidth}
      \begin{block}{Answered Questions}
        How many primes are there?
      \end{block}
    \column{.5\textwidth}
      \begin{block}{Open Questions}
        Is every even number the sum of two primes?
      \end{block}
  \end{columns}
\end{frame}


\begin{frame}[fragile]
  \frametitle{An Algorithm For Finding Primes Numbers.}
\begin{semiverbatim}
\uncover<1->{\alert<0>{int main (void)}}
\uncover<1->{\alert<0>{\{}}
\uncover<1->{\alert<1>{  \alert<4>{std::}vector<bool> is_prime (100, true);}}
\uncover<1->{\alert<1>{  for (int i = 2; i < 100; i++)}}
\uncover<2->{\alert<2>{if (is_prime[i])}}
\uncover<2->{\alert<0>{  \{}}
\uncover<3->{\alert<3>{\}}}
\uncover<3->{\alert<3>{\alert<4>{std::}cout << i << " ";}}
\uncover<3->{\alert<3>{for (int j = i; j < 100;}}
\uncover<2->{\alert<0>{is_prime [j] = false, j+=i);}}
\uncover<1->{\alert<0>{  return 0;}}
\uncover<1->{\alert<0>{\}}}
\end{semiverbatim}
\visible<4->{Note the use of \alert{\texttt{std::}}.}
\end{frame}
\end{document}