\documentclass[11pt,a4paper]{scrartcl}
\usepackage[utf8]{inputenc}
\usepackage[german]{babel}
\usepackage{amsmath}
\usepackage{amsfonts}
\usepackage{amssymb}
\usepackage{graphicx,hyperref}
\usepackage[left=2cm,right=2cm,top=2cm,bottom=2cm]{geometry}
\author{Philipp Arras}

%_______________________________________________________________________________
% Listings
%_______________________________________________________________________________
\usepackage{listings}
\usepackage{color}

\definecolor{mygreen}{rgb}{0,0.6,0}
\definecolor{mygray}{rgb}{0.5,0.5,0.5}
\definecolor{mymauve}{rgb}{0.58,0,0.82}

\lstset{ %
%  backgroundcolor=\color{white},   % choose the background color; you must add \usepackage{color} or \usepackage{xcolor}
basicstyle=\footnotesize,        % the size of the fonts that are used for the codew
breakatwhitespace=false,         % sets if automatic breaks should only happen at whitespace
breaklines=true,                 % sets automatic line breaking
  captionpos=b,                    % sets the caption-position to bottom
  commentstyle=\color{mygreen},    % comment style
%  deletekeywords={...},            % if you want to delete keywords from the given language
%  escapeinside={\%*}{*)},          % if you want to add LaTeX within your code
%  extendedchars=true,              % lets you use non-ASCII characters; for 8-bits encodings only, does not work with UTF-8
  frame=single,                    % adds a frame around the code
%  keepspaces=true,                 % keeps spaces in text, useful for keeping indentation of code (possibly needs columns=flexible)
  keywordstyle=\color{blue},       % keyword style
  language=TeX,                 % the language of the code
%  morekeywords={*,...},            % if you want to add more keywords to the set
  numbers=left,                    % where to put the line-numbers; possible values are (none, left, right)
  numbersep=5pt,                   % how far the line-numbers are from the code
  numberstyle=\tiny\color{mygray}, % the style that is used for the line-numbers
  rulecolor=\color{black},         % if not set, the frame-color may be changed on line-breaks within not-black text (e.g. comments (green here))
%  showspaces=false,                % show spaces everywhere adding particular underscores; it overrides 'showstringspaces'
%  showstringspaces=false,          % underline spaces within strings only
%  showtabs=false,                  % show tabs within strings adding particular underscores
  stepnumber=2,                    % the step between two line-numbers. If it's 1, each line will be numbered
  stringstyle=\color{mymauve},     % string literal style
  tabsize=2,                       % sets default tabsize to 2 spaces
  title=\lstname                   % show the filename of files included with \lstinputlisting; also try caption instead of title
}



\title{\LaTeX -Kurs: 08 Weitere Pakete (Beamer)}
\begin{document}
\maketitle

\begin{itemize}
\item Eigentlich ist \LaTeX\ nicht dazu geeignet Präsentationen oder jegliche Dokumente, wo es auf Design ankommt, zu erstellen. Wenn man jedoch sehr gut strukturierten Inhalt hat, kann man mit der Dokumentenklasse \emph{beamer} auch Bildschirmpräsentationen in Form von PDFs erstellen.
\item Hierfür startet man mit folgendem Template: \lstinputlisting{Beamer-Template.tex}
\item Alle Elemente erklären
  \begin{itemize}
    \item Es gibt unterschiedliche Themes (\url{http://deic.uab.es/~iblanes/beamer_gallery/})
    \item Mit \verb|\titlepage| kann man eine Titelfolie erstellen
    \item Mit Tableofcontents kommt ein Inhaltsverzeichnis. In diesem werden nicht die Folientitel sondern wie in einem normalen Dokument die Abschnitte angezeigt.
    \item Jede Folie besteht aus der Umgebung frame. Als zusätzliches Argument kann ein Folientitel und auch Untertitel ausgegeben werden. Das ist unbedingt empfehlenswert!
    \item Aufbau einer Folie:
    \begin{lstlisting}
\begin{frame}
  \frametitle{There Is No Largest Prime Number}
  \framesubtitle{The proof uses \textit{reductio ad absurdum}.}
  \begin{theorem}
    There is no largest prime number.
  \end{theorem}
  \begin{proof}
    \begin{enumerate}
    \item<1-> Suppose $p$ were the largest prime number.
    \item<2-> Let $q$ be the product of the first $p$ numbers.
    \item<3-> Then $q + 1$ is not divisible by any of them.
    \item<1-> But $q + 1$ is greater than $1$, thus divisible by some prime
      number not in the first $p$ numbers.\qedhere
    \end{enumerate}
\end{proof}
  \uncover<4->{The proof used \textit{reductio ad absurdum}.}
\end{frame}
    \end{lstlisting}
    \item Mit diesem Paket können ganze PDF-Seiten als Folien eingefügt werden.
    \begin{lstlisting}
\usepackage{pdfpages}
{
   \setbeamercolor{background canvas}{bg=}
   \includepdf{somepdfimages.pdf}
}
    \end{lstlisting}
  \end{itemize}	
\item \verb|literature/beamerexample| zeigen
  \begin{itemize}
  \item Die Funktion notes erklären
  \item Titelfolien machen (\verb|\AtBeginSection{\frame{\sectionpage}}|)
  \end{itemize}
\end{itemize}



\end{document}